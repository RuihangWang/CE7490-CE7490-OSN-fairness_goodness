% !TeX root = ../OSN.tex
\subsection{Triadic balance}
This definition is derived directly from
balance theory according to \cite{TB}. The algorithm takes the average product of edge weights for all incomplete triads
that the edge ($u,v$) is a part of. Incomplete triads are triads that would form involving
edge ($u,v$) after it is created. To be specific, let $U_n$ and $V_n$ denotes the set of
neighbors of vertex $u$ and $v$, repectively. To find all possible triads, vertexes with
both connections of $u$ and $v$ are obtained by $N = U_n \cap V_n$. When $N = \emptyset$, 
the weight of ($u,v$) is set to 0. Then, the weight of ($u,v$)
is predicted by:

\begin{equation}
W(u,v) = 
\begin{cases}
\frac{\sum_{n\in N}W(u,n)+W(n,u)+W(v,n)+W(n,v)}{M}, & \text{if $N \neq \emptyset$} \\
0, & \text{otherwise}
\end{cases}
\end{equation}

where M is number of vertexes in set $N$, $W(u,n), W(n,u), W(v,n), W(n,v)$ are weights of each edge connected
to $u$ or $v$ of vertexes in set $N$, repectively.
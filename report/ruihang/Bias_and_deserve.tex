% !TeX root = ../OSN.tex

\subsection{Bias and Deserve}
This method is proposed by Mishra and Bhattacharya in \cite{}.
To compute "bias" and "deserve", we should first normalize the ratings (weights),
and keep them in the range of [-1, 1] where 0 is a neutral opinion. Then, we say node $u$ gives a trust-score
of $w_ij$ to node $v$ for a given rating of ($u, v$). The two attributes of a node are 
defined by:

\begin{itemize}
	\item \emph{Bias}: This reflects the expected weight of an outgoing edge.  
	\item \emph{Deserve}: This reflects the expected weight of an incoming edge from an unbiased vertex.
\end{itemize}

Let $d^o(u)$ denotes the set of all outgoing edges from vertex $u$ and likewise,
$d^i(u)$ denotes the set of all incoming links to node $u$. Then, bias (BIAS) and
deserve (DES) are iteratively computed as:

\begin{equation}
    BIAS^{(t+1)}(u)=\frac{1}{å2|d^o(u)|}\sum_{v \in d^o(u)}[W(u,v) - DES^t(v)]
\end{equation}

\begin{equation}
    DES^{(t+1)}(u)=\frac{1}{å2|d^i(u)|}\sum_{v \in d^i(u)}[W(v,u)(1 - X^t(v,u))]
\end{equation}

where $X^t(v,u) = max\{0, BIAS^t(v) \times W(v,u)\}$. The interative formulations
of bias and deserve allow us to predict the weight of (u,v) based on the deserve
value $DES(v)$ of vertex $v$. Thus, the weight is directly predicted by:

\begin{equation}
    W(u,v) = DES(v)
\end{equation}